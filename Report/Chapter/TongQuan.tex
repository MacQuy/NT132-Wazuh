\chapter{Tổng quan}

\section{Định nghĩa / Giới thiệu}

\textbf{Wazuh} là một nền tảng \textit{mã nguồn mở (open-source)} dùng để \textbf{giám sát an ninh, phát hiện xâm nhập (IDS/IPS)} và \textbf{quản lý thông tin – sự kiện an ninh (SIEM – Security Information and Event Management)}. 

Nó giúp quản trị viên mạng \textbf{phát hiện, phân tích và phản ứng kịp thời} với các hành vi bất thường trong hệ thống, đồng thời hỗ trợ \textbf{kiểm tra tuân thủ (compliance)} theo các tiêu chuẩn bảo mật như PCI DSS, GDPR, HIPAA hay CIS.

Wazuh được phát triển dựa trên dự án \textbf{OSSEC}, nhưng được mở rộng với nhiều cải tiến mạnh mẽ hơn về khả năng trực quan hóa dữ liệu, mở rộng quy mô và quản lý tập trung. Nền tảng này thường được tích hợp với \textbf{Elastic Stack (Elasticsearch, Logstash, Kibana)} để cung cấp \textbf{giao diện giám sát trực quan và phân tích sâu dữ liệu an ninh}.

\subsection*{Mục tiêu của Wazuh}
\begin{itemize}
    \item Phát hiện tấn công và hành vi bất thường trong hệ thống.
    \item Giám sát tính toàn vẹn của tệp tin, phát hiện thay đổi trái phép.
    \item Cung cấp báo cáo tuân thủ an toàn thông tin theo các chuẩn quốc tế.
    \item Phản ứng tự động với sự kiện an ninh (ví dụ: chặn IP, dừng tiến trình, gửi cảnh báo).
\end{itemize}

Nhờ khả năng hoạt động linh hoạt trên nhiều nền tảng như \textbf{Windows, Linux, macOS, container (Docker, Kubernetes)}, Wazuh được sử dụng rộng rãi trong doanh nghiệp, tổ chức chính phủ và môi trường học thuật.

\section{Thành phần}

Một hệ thống Wazuh hoàn chỉnh gồm ba thành phần chính: \textbf{Wazuh Manager}, \textbf{Wazuh Agent}, và \textbf{Wazuh Dashboard (UI)}.

\subsection{Wazuh Manager}
Đây là \textbf{trung tâm xử lý dữ liệu} của toàn hệ thống. Manager nhận log từ các agent, giải mã – phân tích – đối chiếu quy tắc (\textit{rules}) và tạo cảnh báo khi phát hiện bất thường.

Các chức năng chính:
\begin{itemize}
    \item Xử lý và lưu trữ dữ liệu log gửi từ agent.
    \item So khớp các sự kiện với tập quy tắc (ruleset) tích hợp sẵn hoặc do người quản trị tùy chỉnh.
    \item Quản lý danh sách agent, gửi cấu hình và lệnh điều khiển.
    \item Tích hợp với \textbf{Elasticsearch} để lưu trữ dữ liệu và truy vấn nhanh.
\end{itemize}

\subsection{Wazuh Agent}
Agent là \textbf{thành phần được cài trên các máy bị giám sát} (endpoint). Nó thu thập dữ liệu về:
\begin{itemize}
    \item Nhật ký hệ thống (system logs).
    \item Tình trạng tiến trình, registry, hoạt động mạng.
    \item Thay đổi tệp tin quan trọng (file integrity monitoring).
    \item Cấu hình bảo mật của hệ điều hành.
\end{itemize}

Dữ liệu sau đó được mã hóa và gửi về Wazuh Manager thông qua giao thức TCP/UDP an toàn.

\subsection{Wazuh Dashboard (Kibana UI)}
Dashboard cung cấp \textbf{giao diện trực quan} giúp người quản trị theo dõi tình hình an ninh của toàn hệ thống. 

Các tính năng nổi bật:
\begin{itemize}
    \item Biểu đồ thống kê và hiển thị log theo thời gian thực.
    \item Xem chi tiết từng cảnh báo, mức độ rủi ro (low – medium – high – critical).
    \item Quản lý agent, người dùng, rule và cấu hình hệ thống.
    \item Hỗ trợ tìm kiếm và lọc dữ liệu bằng câu lệnh Elasticsearch Query DSL.
\end{itemize}

Ngoài ra, hệ thống có thể tích hợp thêm:
\begin{itemize}
    \item \textbf{Filebeat / Logstash} để truyền dữ liệu log đến Elasticsearch.
    \item \textbf{Elasticsearch} để lưu trữ, phân tích log và cảnh báo.
    \item \textbf{Alerting module} để gửi cảnh báo qua email, Slack, webhook, v.v.
\end{itemize}

\section{Hoạt động}

Cơ chế hoạt động của Wazuh có thể mô tả theo chu trình sau:

\begin{enumerate}
    \item \textbf{Thu thập dữ liệu (Data Collection):} Wazuh Agent thu thập log, trạng thái hệ thống, thay đổi file, tiến trình đang chạy, v.v.
    \item \textbf{Gửi dữ liệu (Data Transmission):} Dữ liệu được mã hóa và gửi về Wazuh Manager thông qua cổng 1514 (TCP hoặc UDP).
    \item \textbf{Phân tích dữ liệu (Data Analysis):} Manager phân tích log dựa trên tập quy tắc (ruleset). Nếu phát hiện hành vi đáng ngờ như brute-force, thay đổi file hệ thống, kết nối trái phép,… thì tạo ra cảnh báo.
    \item \textbf{Lưu trữ và hiển thị (Storage \& Visualization):} Dữ liệu và cảnh báo được lưu vào Elasticsearch. Dashboard hiển thị kết quả trực quan giúp người quản trị dễ dàng nhận biết tình trạng an ninh hiện tại.
    \item \textbf{Phản ứng (Response):} Hệ thống có thể thực hiện hành động tự động như khóa tài khoản, chặn IP, hoặc gửi thông báo khi phát hiện sự cố.
\end{enumerate}