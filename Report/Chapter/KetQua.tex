\chapter{Mô tả quá trình làm việc nhóm}
Chương này viết vế quá trình nhóm chúng mình dựng game trong giai đoạn vừa qua, các kỹ năng và kiến thức chúng mình sử dụng và các thách thức gặp phải cũng như cách mình giải quyết.
\section{Quá trình làm game}
\begin{enumerate}[label=-]
    \item \textbf{28/10/2024:} Phân công nhiệm vụ cho các thành viên.
    \item \textbf{30/10/2024:} Hoàn thành hợp đồng nhóm
    \item \textbf{31/10/2024:} Bắt đầu code SnakeGame theo từng chức năng đã phân công.
    \item \textbf{2/11/2024:} Duy tìm hiểu và áp dụng thư viện "raylib.h".
    \item Phát sinh nhiều lỗi và cả nhóm tìm cách khắc phục và tìm được template. Cả nhóm quyết định sẽ dựa vào template đó để phát triển thêm.
    \item \textbf{10/11/2024:} Duy hoành thành 2 class "Snake and Apple" và push lên GitHub.
    \item \textbf{10/11/2024:} Huy hoàn thành chức ăn Snake "eat and Grow". Dương hoàn thành chức năng "Checking for collision". Dương tiến hành Merge code vào branch chính
    \item  \textbf{11/11/2024:} Duy thêm phần giao diện người dùng "Interface" cho chương trình. Dương Merge vào branch chính
    \item \textbf{12/11/2024: } Dương thêm các chức năng "Add maxscore", và sửa "relogic the main function and fix bug button.cpp".
    \item  \textbf{13/11/2024:} Khoa hoàn thành chức năng "Frame And Score".
    \item \textbf{15/5/2024:} Hoàn thành phần main của game.
    \item \textbf{Thời gian sau đó đến khi nộp đồ án:} Chỉnh sửa thêm về phần đồ họa, hình ảnh.
    \item \textbf{Những khó khăn gặp phải:} 
    \begin{itemize}
        \item Chưa quen với những chức năng cơ bản khi làm việc trên Github. Có các xung đột như merge sai nhánh, tự merge mà không qua trưởng nhóm, code lỗi.
        \item Khó khăn trong việc sử dung thư viện \(raylib\).
        \item Có nhiều ý tưởng khác nhau trong lúc cùng nhau bàn luận cách triển khai code game. Cần xem xét sự hợp lý và làm thử để lựa chọn sao cho phù hợp với kỹ năng và thời gian cả nhóm.
    \end{itemize}
\end{enumerate}

\section{Quá trình viết báo cáo}
Sau khi hoàn thành bài tập SS004.6 và SS004.7, nhóm 11 đã có cơ bản về sử dụng Overleaf. Những thành viên chưa nắm rõ đã tìm hiểu những tài liệu thêm.

\subsection{Hợp đồng nhóm}
\begin{enumerate}[label=-]
    \item \textbf{27/10/2024 - 30/10/2024}: Cả nhóm thông qua hợp đồng và các quy định cần thiết của nhóm và cử ra nhóm trưởng. Sau đó công việc viết hợp đồng nhóm được giao cho Duy và Dương.
    \item \textbf{31/10/2024}: Nhóm tìm được template phù hợp, chỉnh sửa các lỗi trình bày ban đầu và hỗ trợ cho các chương sau.
\end{enumerate}
\subsection{Giới thiệu game}
\begin{enumerate}[label=-]
    \item \textbf{15/11/2024 - 17/11/2024:} Duy và Dương tổng hợp tài liệu và hoàn thiện phần Giới thiệu và Hướng dẫn cách chơi trò chơi. Các thành viên hỗ trợ, góp ý và đánh giá. Và bắt đầu soạn thảo trên Latex.
\end{enumerate}

\subsection{Tài liệu kỹ thuật}
\begin{enumerate}[label=-]
    \item \textbf{15/11/2024 - 17/11/2024:} Khoa và Huy ghi chú, tổng hợp lại các chức năng của từng class cũng như các chức năng của functions trong và ngoài class, Sau đó, hai bạn hoàn thiện phần tài liệu kỹ thuật trò chơi.
\end{enumerate}

\subsection{Mô tả làm việc nhóm}
\begin{enumerate}[label=-]
    \item \textbf{16/11/2024}: Cả nhóm cùng họp và bàn luận để hoàn tất phần mô tả công việc cụ thể của nhóm. Mọi người nêu ra việc đã làm cùng ngày tháng cụ thể để thông tin chính xác, và cùng chỉnh sửa trên Overleaf.
\end{enumerate}

\subsection{Kỹ năng áp dụng}
\begin{enumerate}[label=-]
    \item \textbf{18/11/2024 - 21/11/2024} Tất cả thành viên liệt kê và tổng hợp lại những kĩ năng đã được học và áp dụng khi làm đồ án.
\end{enumerate}

\subsection{Đánh giá}
\begin{enumerate}[label=-]
    \item \textbf{23/11/2024:} Các thành viên họp với nhau để đánh giá các thành viên khác trong đội (đánh giá của cá nhân). Sau đó Nhật tổng hợp lại và hoàn thiện phần đánh giá chung của nhóm.
\end{enumerate}

\subsection{Tài liệu tham khảo}
\begin{enumerate}[label=-]
    \item Trong suốt quá trình làm các công việc trên, cả nhóm cũng ghi chú lại các nội dung mình tham khảo trên mạng để trích nguồn.
    \item Cùng với template hỗ trợ sẵn, các thành viên trích ra nguồn mình đã sử dụng, đảm bảo tính minh bạch rõ ràng cho báo cáo.
\end{enumerate}
